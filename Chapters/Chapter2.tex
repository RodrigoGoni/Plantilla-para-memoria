% --- Esqueleto para el Capítulo 2 ---

\chapter{Introducción específica}
\label{Chapter2}

% Párrafo introductorio del capítulo.

En este capítulo se presenta el marco teórico que sustentó el desarrollo del
trabajo. Se detallan las tecnologías de \textit{software} seleccionadas para el
análisis y la optimización, y las métricas definidas para la validación del
sistema. El contenido de este capítulo se basa en la planificación y los
conceptos establecidos en el documento de gestión del proyecto
\citep{RGv5Goni}. % Ejemplo de cita

% ------------------------------------------

\section{Marco teórico}
\label{sec:marco_teorico}

% Tono: Presente (describe la teoría).
El desarrollo del sistema se fundamentó en dos pilares tecnológicos
principales: la visión artificial para el análisis cinemático y los algoritmos
genéticos para la optimización.

\subsection{Visión artificial para estimación de pose}
\label{ssec:vision_pose}

% Tono: Pasado (lo que se investigó/hizo).
Para la extracción de las coordenadas 2D de los puntos clave del ciclista desde
la captura de video, se realizó un análisis de las herramientas de estimación
de pose (\textit{pose estimation}) disponibles. Se evaluaron modelos
preentrenados como Keypoint R-CNN \citep{maskrcnn} y MediaPipe \citep{MediaPipe}, mencionados en la planificación
del proyecto \citep{RGv5Goni}.

% Tono: Presente (teoría).
Estos modelos utilizan redes neuronales convolucionales profundas (CNN) para
identificar y localizar articulaciones humanas clave (ej. codos, rodillas,
tobillos) en cada fotograma. Las CNN se han convertido en el estándar en tareas
de visión por computador, con un diseño inspirado en el córtex visual humano.
Su principio fundamental es la abstracción jerárquica de características
\citep{oshea2015introduction}. Las capas iniciales detectan rasgos de bajo
nivel como bordes y texturas, que se componen en representaciones más complejas
en capas sucesivas.

El paradigma dominante actual para esta tarea no es la regresión directa de
coordenadas, la que demostró ser difícil, particularmente ante oclusiones
\citep{ar5iv_heatmap_regression}. En su lugar, el problema se reformula como la
detección basada en mapas de calor (\textit{heatmaps})
\citep{medium_vanishing_point_heatmap}. En este enfoque, la CNN genera un
conjunto de mapas de calor 2D, uno por cada articulación
\citep{stackoverflow_parse_heatmap}. Cada mapa de calor funciona como una
distribución de probabilidad espacial, donde el valor de cada píxel representa
la confianza de que la articulación se ubique allí. Este método permite
codificar la incertidumbre: una articulación visible genera un pico de alta
confianza, mientras que una articulación oculta se representa por un pico más
difuso y bajo.

% Tono: Pasado (la decisión tomada).
Finalmente, se seleccionó el modelo \texttt{MediaPipe BlazePose}. Esta decisión
no se basó únicamente en la precisión, sino en dos justificaciones de dominio
específicas. En primer lugar, su alineación con la tarea: BlazePose es un
rastreador (\textit{tracker}) optimizado para la coherencia temporal en
vídeo \citep{learnopencv_yolov7_vs_mediapipe}, a diferencia de los
modelos que detectan en cada fotograma. En segundo lugar, su alineación con el
dominio biomecánico: el modelo predice una topología de 33 \textit{landmarks}
3D \citep{googleai_pose_landmarker} que incluye puntos clave en pies y talones,
elementos indispensables para el análisis biomecánico del ciclismo que no están
presentes en las 17 articulaciones de los modelos COCO
\citep{learnopencv_keypoint_rcnn}.

% Tono: Presente (descripción de la tabla).
En la tabla \ref{tab:comparativa_modelos} se presenta una comparativa de los
modelos evaluados.

% ----- Placeholder: Tabla Comparativa -----
\begin{table}[htbp]
      \caption{Tabla comparativa de modelos de estimación de pose.}
      \label{tab:comparativa_modelos}
      \centering
      \small
      \begin{tabular}{p{2.5cm}p{2.2cm}p{2.2cm}p{5.5cm}}
            \toprule
            \textbf{Modelo}     & \textbf{mAP}                                          & \textbf{FPS}                                                & \textbf{Observaciones}                                                                                              \\
            \midrule
            MediaPipe BlazePose & 62.6--68.1 en 17 kpts \citep{ar5iv_blazepose_ghum} & $\sim$30 FPS CPU \citep{learnopencv_yolov7_vs_mediapipe} & Optimizado para \textit{tracking} en video. Topología de 33 \textit{landmarks} 3D \citep{googleai_pose_landmarker}. \\
            \midrule
            Keypoint R-CNN      & 63.1 kp. AP en COCO \citep{maskrcnn} & $\sim$5 FPS \citep{maskrcnn}  & Alta precisión. Costo escala linealmente con número de personas \citep{learnopencv_yolov7_vs_mediapipe}.             \\
            \bottomrule
      \end{tabular}
\end{table}
% -------------------------------------------

\subsection{Algoritmos genéticos}
\label{ssec:algoritmos-geneticos}

% Tono: Presente (teoría).
Los algoritmos genéticos (AG, \textit{Genetic Algorithms}) son metaheurísticas
de búsqueda inspiradas en el proceso de selección natural \citep{goldberg1989}. Pertenecen a la clase más amplia de algoritmos evolutivos y
son particularmente efectivos para problemas de optimización complejos,
multimodales y no lineales, donde los métodos determinísticos tradicionales
pueden fallar \citep{goldberg1989}.

El funcionamiento de un algoritmo genético se basa en la evolución de una
población de soluciones candidatas (individuos) durante varias generaciones
\citep{goldberg1989}. Cada individuo representa una solución potencial
completa al problema y se codifica en un cromosoma \citep{goldberg1989}. Para problemas de ingeniería, se utiliza comúnmente una
codificación de valor real.

El componente más crítico es la función de aptitud (\textit{fitness function}),
que es la única conexión del algoritmo con el problema de dominio. Esta función
toma un individuo como entrada y devuelve un valor escalar que cuantifica la
calidad de esa solución \citep{goldberg1989}. El objetivo del algoritmo
genético es maximizar (o minimizar) este valor.

La evolución de la población se logra mediante la aplicación iterativa de
operadores genéticos \citep{goldberg1989}:
\begin{itemize}
      \item Selección: se determina qué individuos de la población actual se reproducirán,
            con base en el principio de que los individuos con mayor aptitud deben tener
            una mayor probabilidad de ser seleccionados \citep{goldberg1989}. Un
            método robusto y común es la selección por torneo
            \citep{youtube_seleccion_individuos, wikipedia_seleccion_torneos}.
      \item Cruce (\textit{crossover}): es el operador de explotación. Combina la
            información genética de dos individuos parentales para crear descendencia, con
            el objetivo de heredar rasgos beneficiosos y converger en regiones prometedoras
            del espacio de búsqueda \citep{goldberg1989}.
      \item Mutación: es el operador de exploración. Introduce cambios pequeños y
            aleatorios en el cromosoma de un individuo \citep{goldberg1989}. Es
            esencial para mantener la diversidad genética y permitir que el algoritmo
            escape de óptimos locales \citep{goldberg1989}.
\end{itemize}
El poder del algoritmo genético reside en el equilibrio entre la explotación (cruce) y la exploración (mutación), lo que previene la convergencia prematura a soluciones subóptimas \citep{goldberg1989}.

% Tono: Pasado (aplicación en el trabajo).
En este trabajo, se utilizó un algoritmo genético para explorar el espacio de
soluciones de los parámetros biomecánicos, como se detalla en la sección
\ref{sec:optimizacion}.
\section{Algoritmos de optimización}
\label{sec:optimizacion}

% Tono: Presente (definición del problema).
El problema central del trabajo se definió como un problema de optimización
multiobjetivo (MOO, \textit{Multi-Objective Optimization}). A diferencia de la
optimización mono-objetivo, la MOO busca optimizar simultáneamente dos o más
funciones objetivo que, a menudo, están en conflicto.

% Tono: Pasado (aplicación en el trabajo).
En el contexto de este trabajo, los objetivos en conflicto fueron la
maximización de la eficiencia biomecánica y la minimización de la resistencia
aerodinámica, sujetos a un conjunto de restricciones biomecánicas para prevenir
lesiones.

% Tono: Presente (teoría).
La solución a un problema MOO no es un único valor óptimo, sino un conjunto de
soluciones óptimas de compromiso, conocido como el frente de Pareto. Cada
solución en el frente de Pareto es óptima en el sentido de que no se puede
mejorar un objetivo sin empeorar al menos otro.

% Tono: Presente (descripción de la figura).
La figura \ref{fig:pareto_conceptual} ilustra el concepto de la frontera de
Pareto en el contexto de los objetivos del trabajo. La superficie coloreada
representa la frontera (envolvente convexa) que conecta todas las soluciones
Pareto-óptimas, formando el límite del espacio de soluciones factibles. Los
puntos negros sobre la superficie son las soluciones específicas evaluadas por
el algoritmo genético. El punto dorado (diamante) indica la solución de
compromiso recomendada, que balancea los tres objetivos simultáneamente,
mientras que los puntos cuadrados (azul, rojo y verde) muestran las soluciones
que optimizan cada objetivo de forma individual.

% ----- Figura: Frontera de Pareto 3D -----
\begin{figure}[H]
      \centering
      \includegraphics[width=0.85\textwidth]{./Figures/pareto_clean.pdf}
      \caption{Frontera de Pareto tridimensional con los tres objetivos a minimizar: coeficiente aerodinámico (CdA), frecuencia cardíaca y riesgo de lesión.}
      \label{fig:pareto_conceptual}
\end{figure}
% ---------------------------------------

\section{Métricas de evaluación}
\label{sec:metricas}

% Tono: Pasado (se definieron).
Para evaluar la efectividad de las configuraciones recomendadas por el sistema,
se definió un conjunto de métricas objetivas y subjetivas, basadas en la
sección 5 (evaluación del modelo) de la planificación \citep{RGv5Goni}.

\subsection{Métricas objetivas}
\label{ssec:metricas_objetivas}

% Tono: Presente (definiciones).
Las métricas objetivas son mediciones cuantitativas directas obtenidas de los
sensores y del análisis cinemático.

\begin{itemize}
      \item Eficiencia cardiovascular (FC): se define como la reducción de la frecuencia
            cardíaca (FC) media para un bloque de potencia constante y duración definida.
            Una FC menor para el mismo esfuerzo indica una mayor eficiencia. Se consideran
            las zonas cardíacas como referencia (Z1-Z5) para contextualizar el esfuerzo.

      \item Potencia (W): medición de la potencia de salida sostenida durante el intervalo
            de análisis. Se registra la potencia media, máxima y normalizada, considerando
            las zonas de potencia funcionales (FTP, \textit{Functional Threshold Power})
            del ciclista para evaluar la intensidad del esfuerzo.

      \item Cadencia (RPM): frecuencia de pedaleo expresada en revoluciones por minuto. Una
            cadencia óptima suele estar entre 80--100 RPM para la mayoría de ciclistas,
            aunque depende del perfil individual.

      \item Velocidad (km/h): velocidad de desplazamiento medida durante el test en
            condiciones controladas (ej. rodillo inteligente o circuito cerrado).

      \item Balance de potencia L/R (\%): distribución de la potencia generada entre la
            pierna izquierda y derecha. Un balance óptimo es cercano a 50\%/50\%, aunque
            desbalances menores al 5\% son considerados normales. Desbalances mayores
            pueden indicar asimetrías biomecánicas o riesgo de lesión.

      \item Fase de potencia (\textit{Power Phase}): rango angular del ciclo de pedaleo
            donde cada pierna genera potencia positiva. Se mide en grados y permite
            identificar la eficiencia de la aplicación de fuerza durante el pedaleo.

      \item Desfase de fase de potencia (\textit{Power Phase Offset}): diferencia angular
            entre el inicio de la fase de potencia de cada pierna. Valores elevados pueden
            indicar asimetrías en el patrón de pedaleo.

      \item Variabilidad cinemática: se define como la desviación estándar de ángulos
            articulares clave (ej. extensión de rodilla, flexión de cadera) durante
            múltiples ciclos de pedaleo. Una menor variabilidad sugiere un movimiento más
            estable, eficiente y con menor riesgo de lesiones por sobrecarga.
\end{itemize}

\subsection{Métricas subjetivas}
\label{ssec:metricas_subjetivas}

% Tono: Presente (definiciones).
Las métricas subjetivas cuantifican la percepción del ciclista, lo que es
fundamental para validar el confort.

\begin{itemize}
      \item Percepción del esfuerzo (RPE, \textit{Rating of Perceived Exertion}): se
            utiliza la escala de Borg de 6 a 20. Se solicita al ciclista que valore su
            esfuerzo percibido en condiciones controladas (misma potencia y duración) antes
            y después del ajuste.
      \item Escalas de confort/molestia: un reporte cualitativo o en escala de 1 a 10 sobre
            la presencia de molestias en zonas específicas (rodillas, espalda baja,
            hombros, manos).
\end{itemize}

% Tono: Presente (descripción de la tabla).
En la tabla \ref{tab:metricas_definicion} se resumen las métricas utilizadas
para el proceso de validación.

% ----- Placeholder: Tabla Métricas -----
\begin{table}[htbp]
      \caption{Tabla de definición de métricas de evaluación.}
      \label{tab:metricas_definicion}
      \centering
      \small
      \begin{tabular}{p{2cm}p{3cm}p{6cm}p{1.8cm}}
            \toprule
            \textbf{Tipo} & \textbf{Métrica}        & \textbf{Definición}                       & \textbf{Unidad} \\
            \midrule
            Objetiva      & FC (eficiencia)         & FC media en zonas de potencia controladas & PPM             \\
            Objetiva      & Potencia                & Potencia media/normalizada en zona FTP    & W               \\
            Objetiva      & Cadencia                & Frecuencia de pedaleo media               & RPM             \\
            Objetiva      & Velocidad               & Velocidad en condiciones controladas      & km/h            \\
            Objetiva      & Balance L/R             & Distribución de potencia entre piernas    & \%              \\
            Objetiva      & Fase de potencia        & Rango angular de aplicación de potencia   & Grados          \\
            Objetiva      & Desfase de fase         & Diferencia angular entre piernas          & Grados          \\
            Objetiva      & Variabilidad cinemática & Desv. estándar de ángulos articulares    & Grados          \\
            \midrule
            Subjetiva     & RPE                     & Escala de Borg (6--20)                    & Adim.           \\
            Subjetiva     & Confort                 & Escala de dolor (0--10)                   & Adim.           \\
            \bottomrule
      \end{tabular}
\end{table}
% ---------------------------------------

% --- Fin del Capítulo 2 ---