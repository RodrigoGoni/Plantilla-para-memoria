% --- Esqueleto para el Capítulo 2 ---


\chapter{Introducción específica}
\label{Chapter2}

% Párrafo introductorio del capítulo.

En este capítulo se presenta el marco metodológico y teórico que sustentó el desarrollo del trabajo. Se detallan las fases del modelo de gestión adaptado, las tecnologías de \textit{software} seleccionadas para el análisis y la optimización, y las métricas definidas para la validación del sistema. 
El contenido de este capítulo se basa en la planificación y los conceptos establecidos en el documento de gestión del proyecto \citep{RGv5Goni}. % Ejemplo de cita

\section{Metodología}
\label{sec:metodologia}

% Tono: Pasado (describe lo que SE HIZO en el proyecto).
Para la gestión y ejecución de este trabajo, se adaptó la metodología CRISP-DM (\textit{Cross-Industry Standard Process for Data Mining}). Esta adaptación fue necesaria para alinear las fases estándar de un proyecto de minería de datos con los objetivos específicos de un proyecto de optimización biomecánica asistida por inteligencia artificial, como se describe en la sección 8 del documento de planificación \citep{RGv5Goni}.

% Tono: Pasado (el trabajo de adaptación).
El modelo se reestructuró en las siguientes fases principales, ajustadas al contexto del trabajo:

\begin{itemize}
    \item Comprensión del negocio y del problema biomecánico.
    \item Comprensión y adquisición de los datos (video, sensores de rendimiento y datos antropométricos).
    \item Preparación y preprocesamiento de los datos, incluyendo la extracción de pose y el filtrado de señales.
    \item Modelado y simulación biomecánica y aerodinámica.
    \item Evaluación de las soluciones generadas por el optimizador (análisis del frente de Pareto).
    \item Despliegue, materializado en la generación del reporte de recomendaciones para el ciclista.
\end{itemize}

La figura \ref{fig:crisp_dm_flujo} muestra el diagrama de flujo de la metodología implementada.

\begin{figure}[htb!]
    \centering
    % \includegraphics[width=0.8\textwidth]{Imagenes/crisp_dm.png} % Descomentar para usar la imagen real
    
    % --- Inicio: Placeholder de la figura ---
    % Se utiliza \framebox para un placeholder robusto.
	\includegraphics[width=0.7\textwidth]{./Figures/crispdm.pdf}
    % --- Fin: Placeholder de la figura ---
    
    \caption{Diagrama de flujo de la metodología CRISP-DM adaptada para el trabajo.}
    \label{fig:crisp_dm_flujo}
\end{figure}
% ------------------------------------------

\section{Marco teórico}
\label{sec:marco_teorico}

% Tono: Presente (describe la teoría).
El desarrollo del sistema se fundamentó en dos pilares tecnológicos principales: la visión artificial para el análisis cinemático y los algoritmos genéticos para la optimización.

\subsection{Visión artificial para estimación de pose}
\label{ssec:vision_pose}

% Tono: Pasado (lo que se investigó/hizo).
Para la extracción de las coordenadas 2D de los puntos clave del ciclista desde la captura de video, se realizó un análisis de las herramientas de estimación de pose (\textit{pose estimation}) disponibles. Se evaluaron modelos preentrenados como Keypoint R-CNN y MediaPipe, mencionados en la planificación del proyecto \citep{RGv5Goni}.

% Tono: Presente (teoría).
Estos modelos utilizan redes neuronales convolucionales profundas (CNN) para identificar y localizar articulaciones humanas clave (ej. codos, rodillas, tobillos) en cada fotograma. Las CNN se han convertido en el estándar en tareas de visión por computador, con un diseño inspirado en el córtex visual humano. Su principio fundamental es la abstracción jerárquica de características \citep{oshea2015introduction}: las capas iniciales detectan rasgos de bajo nivel como bordes y texturas, que se componen en representaciones más complejas en capas sucesivas.

El paradigma dominante actual para esta tarea no es la regresión directa de coordenadas, la cual demostró ser difícil, particularmente ante oclusiones \citep{ar5iv_heatmap_regression}. En su lugar, el problema se reformula como la detección basada en mapas de calor (\textit{heatmaps}) \citep{medium_vanishing_point_heatmap}. En este enfoque, la CNN genera un conjunto de mapas de calor 2D, uno por cada articulación \citep{stackoverflow_parse_heatmap}. Cada mapa de calor funciona como una distribución de probabilidad espacial, donde el valor de cada píxel representa la confianza de que la articulación se ubique allí. Este método permite codificar la incertidumbre: una articulación visible genera un pico de alta confianza, mientras que una articulación oculta se representa por un pico más difuso y bajo.

% Tono: Pasado (la decisión tomada).
Finalmente, se seleccionó el modelo \textit{MediaPipe BlazePose}. Esta decisión no se basó únicamente en la precisión, sino en dos justificaciones de dominio específicas. Primero, su alineación con la tarea: BlazePose es un rastreador (\textit{tracker}) optimizado para la coherencia temporal en \textit{vídeo} \citep{learnopencv_yolov7_vs_mediapipe}, a diferencia de los modelos que detectan en cada fotograma. Segundo, su alineación con el dominio biomecánico: el modelo predice una topología de 33 \textit{landmarks} 3D \citep{googleai_pose_landmarker} que incluye puntos clave en pies y talones, elementos indispensables para el análisis biomecánico del ciclismo que no están presentes en las 17 articulaciones de los modelos COCO \citep{learnopencv_keypoint_rcnn}.

% Tono: Presente (descripción de la tabla).
En la tabla \ref{tab:comparativa_modelos} se presenta una comparativa de los modelos evaluados.

% ----- Placeholder: Tabla Comparativa -----
\begin{table}[htbp]
    \caption{Tabla comparativa de modelos de estimación de pose.}
    \label{tab:comparativa_modelos}
    \centering
    \small
    \begin{tabular}{p{3cm}p{3cm}p{2.5cm}p{5cm}}
        \toprule
        \textbf{Modelo} & \textbf{Precisión (mAP)} & \textbf{Velocidad (FPS)} & \textbf{Observaciones} \\
        \midrule
        MediaPipe BlazePose & 62.6 - 68.1 (en 17 kpts) \citep{ar5iv_blazepose_ghum} & \textasciitilde{}30 FPS (CPU Laptop) \citep{learnopencv_yolov7_vs_mediapipe} & Optimizado para \textit{tracking} en video. Topología de 33 \textit{landmarks} 3D \citep{googleai_pose_landmarker}. \\
        \midrule
        Keypoint R-CNN & 64.0 (kp. AP en COCO) \citep{detectrontest_model_zoo} & \textasciitilde{}13.8 FPS (GPU V100) \citep{detectrontest_model_zoo} & Alta precisión. Costo escala linealmente con Nro. de personas \citep{learnopencv_yolov7_vs_mediapipe}. \\
        \bottomrule
    \end{tabular}
\end{table}
% -------------------------------------------

\subsection{Algoritmos Genéticos}
\label{ssec:algoritmos-geneticos}

% Tono: Presente (teoría).
Los algoritmos genéticos (AG) son metaheurísticas de búsqueda inspiradas en el proceso de selección natural \citep{adictosaltrabajo_jgap, ceca_uaeh_algoritmos_geneticos}. Pertenecen a la clase más amplia de algoritmos evolutivos y son particularmente efectivos para problemas de optimización complejos, multimodales y no lineales, donde los métodos determinísticos tradicionales pueden fallar.

El funcionamiento de un AG se basa en la evolución de una población de soluciones candidatas (individuos) durante varias generaciones. Cada individuo representa una solución potencial completa al problema y se codifica en un cromosoma. Para problemas de ingeniería, se utiliza comúnmente una codificación de valor real.

El componente más crítico es la función de aptitud (\textit{fitness function}), que es la única conexión del algoritmo con el problema de dominio. Esta función toma un individuo como entrada y devuelve un valor escalar que cuantifica la calidad de esa solución. El objetivo del AG es maximizar (o minimizar) este valor.

La evolución de la población se logra mediante la aplicación iterativa de operadores genéticos:
\begin{itemize}
    \item \textbf{Selección:} Determina qué individuos de la población actual se reproducirán, basándose en el principio de que los individuos con mayor aptitud deben tener una mayor probabilidad de ser seleccionados. Un método robusto y común es la selección por torneo \citep{youtube_seleccion_individuos, wikipedia_seleccion_torneos}.
    \item \textbf{Cruce (\textit{Crossover}):} Es el operador de explotación. Combina la información genética de dos individuos parentales para crear descendencia, con el objetivo de heredar rasgos beneficiosos y converger en regiones prometedoras del espacio de búsqueda.
    \item \textbf{Mutación:} Es el operador de exploración. Introduce cambios pequeños y aleatorios en el cromosoma de un individuo. Es esencial para mantener la diversidad genética y permitir que el algoritmo escape de óptimos locales.
\end{itemize}
El poder del AG reside en el equilibrio entre la explotación (cruce) y la exploración (mutación), previniendo la convergencia prematura a soluciones subóptimas \citep{datacamp_genetic_algorithm_python, sedici_algoritmos_evolutivos_avanzados, wikipedia_algoritmo_genetico}.

% Tono: Pasado (aplicación en el proyecto).
En este proyecto, se utilizó un AG para explorar el espacio de soluciones de los parámetros biomecánicos, como se detalla en la sección \ref{sec:optimizacion}.
\section{Algoritmos de optimización}
\label{sec:optimizacion}

% Tono: Presente (definición del problema).
El problema central del proyecto se define como un problema de optimización multiobjetivo (MOO). A diferencia de la optimización mono-objetivo, la MOO busca optimizar simultáneamente dos o más funciones objetivo que, a menudo, están en conflicto.

% Tono: Pasado (aplicación en el proyecto).
En el contexto de este trabajo, los objetivos en conflicto fueron la maximización de la eficiencia biomecánica y la minimización de la resistencia aerodinámica, sujeto a un conjunto de restricciones biomecánicas para prevenir lesiones.

% Tono: Presente (teoría).
La solución a un problema MOO no es un único valor óptimo, sino un conjunto de soluciones óptimas de compromiso, conocido como el frente de Pareto. Cada solución en el frente de Pareto es óptima en el sentido de que no se puede mejorar un objetivo sin empeorar al menos otro.

% Tono: Presente (descripción de la figura).
La figura \ref{fig:pareto_conceptual} ilustra el concepto de la frontera de Pareto en el contexto de los objetivos del proyecto. La superficie coloreada representa la frontera (envolvente convexa) que conecta todas las soluciones Pareto-óptimas, formando el límite del espacio de soluciones factibles. Los puntos negros sobre la superficie son las soluciones específicas evaluadas por el algoritmo genético. El punto dorado (diamante) indica la solución de compromiso recomendada, que balancea los tres objetivos simultáneamente, mientras que los puntos cuadrados (azul, rojo y verde) muestran las soluciones que optimizan cada objetivo de forma individual.

% ----- Figura: Frontera de Pareto 3D -----
\begin{figure}[htb!]
    \centering
    \includegraphics[width=0.85\textwidth]{./Figures/pareto_clean.pdf}
    \caption{Frontera de Pareto tridimensional con los tres objetivos a minimizar: coeficiente aerodinámico (CdA), frecuencia cardíaca y riesgo de lesión.}
    \label{fig:pareto_conceptual}
\end{figure}
% ---------------------------------------

\section{Métricas de evaluación}
\label{sec:metricas}

% Tono: Pasado (se definieron).
Para evaluar la efectividad de las configuraciones recomendadas por el sistema, se definió un conjunto de métricas objetivas y subjetivas, basadas en la sección 5 (Evaluación del modelo) de la planificación \citep{RGv5Goni}.

\subsection{Métricas objetivas}
\label{ssec:metricas_objetivas}

% Tono: Presente (definiciones).
Las métricas objetivas son mediciones cuantitativas directas obtenidas de los sensores y del análisis cinemático.

\begin{itemize}
    \item \textbf{Eficiencia cardiovascular (FC):} Se define como la reducción de la frecuencia cardíaca (FC) media para un bloque de potencia constante y duración definida. Una FC menor para el mismo esfuerzo indica una mayor eficiencia. Se consideran las zonas cardíacas como referencia (Z1-Z5) para contextualizar el esfuerzo.
    
    \item \textbf{Potencia (W):} Medición de la potencia de salida sostenida durante el intervalo de análisis. Se registra la potencia media, máxima y normalizada, considerando las zonas de potencia funcionales (FTP) del ciclista para evaluar la intensidad del esfuerzo.
    
    \item \textbf{Cadencia (RPM):} Frecuencia de pedaleo expresada en revoluciones por minuto. Una cadencia óptima suele estar entre 80-100 RPM para la mayoría de ciclistas, aunque depende del perfil individual.
    
    \item \textbf{Velocidad (km/h):} Velocidad de desplazamiento medida durante el test en condiciones controladas (ej. rodillo inteligente o circuito cerrado).
    
    \item \textbf{Balance de potencia L/R (\%):} Distribución de la potencia generada entre la pierna izquierda y derecha. Un balance óptimo es cercano a 50\%/50\%, aunque desbalances menores al 5\% son considerados normales. Desbalances mayores pueden indicar asimetrías biomecánicas o riesgo de lesión.
    
    \item \textbf{Fase de potencia (\textit{Power Phase}):} Rango angular del ciclo de pedaleo donde cada pierna genera potencia positiva. Se mide en grados y permite identificar la eficiencia de la aplicación de fuerza durante el pedaleo.
    
    \item \textbf{Desfase de fase de potencia (\textit{Power Phase Offset}):} Diferencia angular entre el inicio de la fase de potencia de cada pierna. Valores elevados pueden indicar asimetrías en el patrón de pedaleo.
    
    \item \textbf{Variabilidad cinemática:} Se define como la desviación estándar de ángulos articulares clave (ej. extensión de rodilla, flexión de cadera) durante múltiples ciclos de pedaleo. Una menor variabilidad sugiere un movimiento más estable, eficiente y con menor riesgo de lesiones por sobrecarga.
\end{itemize}

\subsection{Métricas subjetivas}
\label{ssec:metricas_subjetivas}

% Tono: Presente (definiciones).
Las métricas subjetivas cuantifican la percepción del ciclista, lo cual es fundamental para validar el confort.

\begin{itemize}
    \item \textbf{Percepción del Esfuerzo (RPE):} Se utiliza la escala de Borg (RPE, \textit{Rating of Perceived Exertion}) de 6 a 20. Se solicita al ciclista que valore su esfuerzo percibido en condiciones controladas (misma potencia y duración) antes y después del ajuste.
    \item \textbf{Escalas de Confort/Molestia:} Un reporte cualitativo o en escala de 1 a 10 sobre la presencia de molestias en zonas específicas (rodillas, espalda baja, hombros, manos).
\end{itemize}

% Tono: Presente (descripción de la tabla).
En la tabla \ref{tab:metricas_definicion} se resumen las métricas utilizadas para el proceso de validación.

% ----- Placeholder: Tabla Métricas -----
\begin{table}[htbp]
    \caption{Tabla de definición de métricas de evaluación.}
    \label{tab:metricas_definicion}
    \centering
    \small
    \begin{tabular}{p{2.5cm}p{3.5cm}p{5.5cm}p{2cm}}
        \toprule
        \textbf{Tipo} & \textbf{Métrica} & \textbf{Definición} & \textbf{Unidad} \\
        \midrule
        Objetiva & FC (eficiencia) & FC media en zonas de potencia controladas & PPM \\
        Objetiva & Potencia & Potencia media/normalizada en zona FTP & W \\
        Objetiva & Cadencia & Frecuencia de pedaleo media & RPM \\
        Objetiva & Velocidad & Velocidad en condiciones controladas & km/h \\
        Objetiva & Balance L/R & Distribución de potencia entre piernas & \% \\
        Objetiva & Fase de potencia & Rango angular de aplicación de potencia & Grados (º) \\
        Objetiva & Desfase de fase & Diferencia angular entre piernas & Grados (º) \\
        Objetiva & Variabilidad cinemática & Desv. estándar de ángulos articulares & Grados (º) \\
        \midrule
        Subjetiva & RPE & Escala de Borg (6--20) & Adimensional \\
        Subjetiva & Confort & Escala de dolor (0--10) & Adimensional \\
        \bottomrule
    \end{tabular}
\end{table}
% ---------------------------------------

% --- Fin del Capítulo 2 ---