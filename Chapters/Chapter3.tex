\chapter{Diseño e implementación} % Main chapter title

\label{Chapter3} % Change X to a consecutive number; for referencing this chapter elsewhere, use \ref{ChapterX}

Todos los capítulos deben comenzar con un breve párrafo introductorio que indique cuál es el contenido que se encontrará al leerlo.  La redacción sobre el contenido de la memoria debe hacerse en presente y todo lo referido al proyecto en pasado, siempre de modo impersonal.

%----------------------------------------------------------------------------------------
%	METODOLOGÍA
%----------------------------------------------------------------------------------------
\section{Metodología}
\label{sec:metodologia}

% Tono: Pasado (describe lo que SE HIZO en el trabajo).
Para la gestión y ejecución de este trabajo, se adaptó la metodología CRISP-DM \citep{IBM_SPSS_CRISP_DM}
(\textit{Cross-Industry Standard Process for Data Mining}). Esta adaptación fue
necesaria para alinear las fases estándar de un proyecto de minería de datos
con los objetivos específicos del trabajo de optimización biomecánica
asistida por inteligencia artificial, como se describe en la sección 8 del
documento de planificación \citep{RGv5Goni}.

% Tono: Pasado (el trabajo de adaptación).
El modelo se reestructuró en las siguientes fases principales, ajustadas al
contexto del trabajo:

\begin{itemize}
    \item Comprensión del negocio y del problema biomecánico.
    \item Comprensión y adquisición de los datos (video, sensores de rendimiento y datos
          antropométricos).
    \item Preparación y preprocesamiento de los datos, que incluyó la extracción de pose y
          el filtrado de señales.
    \item Modelado y simulación biomecánica y aerodinámica.
    \item Evaluación de las soluciones generadas por el optimizador (análisis del frente
          de Pareto).
    \item Despliegue, que fue materializado en la generación del reporte de recomendaciones para
          el ciclista.
\end{itemize}

La figura \ref{fig:crisp_dm_flujo} muestra el diagrama de flujo de la
metodología implementada.

\begin{figure}[H]
    \centering
    % \includegraphics[width=0.8\textwidth]{Imagenes/crisp_dm.png} % Descomentar para usar la imagen real

    % --- Inicio: Placeholder de la figura ---
    % Se utiliza \framebox para un placeholder robusto.
    \includegraphics[width=0.7\textwidth]{./Figures/crispdm.pdf}
    % --- Fin: Placeholder de la figura ---

    \caption{Diagrama de flujo de la metodología CRISP-DM adaptada para el trabajo.}
    \label{fig:crisp_dm_flujo}
\end{figure}

%----------------------------------------------------------------------------------------
%	CONFIGURACIÓN DE CÓDIGO
%----------------------------------------------------------------------------------------



