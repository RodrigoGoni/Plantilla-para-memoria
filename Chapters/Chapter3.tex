\chapter{Diseño e implementación}
\label{Chapter3}

En este capítulo se presenta el diseño e implementación del sistema de
optimización biomecánica desarrollado en este trabajo. Se describe la
metodología empleada para estructurar el trabajo, la arquitectura del sistema
propuesto y los módulos que lo componen. La exposición se organiza de manera
secuencial, desde la adquisición de datos hasta la generación del reporte final
de recomendaciones.

%----------------------------------------------------------------------------------------
%	METODOLOGÍA
%----------------------------------------------------------------------------------------
\section{Metodología}
\label{sec:metodologia}

% Tono: Pasado (describe lo que SE HIZO en el trabajo).
Para la gestión y ejecución de este trabajo, se adaptó la metodología CRISP-DM
\citep{IBM_SPSS_CRISP_DM} (\textit{Cross-Industry Standard Process for Data
  Mining}). Esta adaptación fue necesaria para alinear las fases estándar de un
trabajo de minería de datos con los objetivos específicos del trabajo de
optimización biomecánica asistida por inteligencia artificial, como se describe
en la sección 8 del documento de planificación \citep{RGv5Goni}.

% Tono: Pasado (el trabajo de adaptación).
El modelo se reestructuró en las siguientes fases principales, que fueron
ajustadas al contexto del trabajo:

\begin{itemize}
  \item Comprensión del negocio y del problema biomecánico.
  \item Comprensión y adquisición de los datos (video, sensores de rendimiento y datos
        antropométricos).
  \item Preparación y preprocesamiento de los datos, que incluyó la extracción de pose
        y el filtrado de señales.
  \item Modelado y simulación biomecánica y aerodinámica.
  \item Evaluación de las soluciones generadas por el optimizador (análisis del frente
        de Pareto).
  \item Despliegue, que fue materializado en la generación del reporte de
        recomendaciones para el ciclista.
\end{itemize}

La figura \ref{fig:crisp_dm_flujo} muestra el diagrama de flujo de la
metodología implementada.

\begin{figure}[H]
  \centering
  % \includegraphics[width=0.8\textwidth]{Imagenes/crisp_dm.png} % Descomentar para usar la imagen real

  % --- Inicio: Placeholder de la figura ---
  % Se utiliza \framebox para un placeholder robusto.
  \includegraphics[width=0.7\textwidth]{./Figures/crispdm.pdf}
  % --- Fin: Placeholder de la figura ---

  \caption{Diagrama de flujo de la metodología CRISP-DM adaptada para el trabajo.}
  \label{fig:crisp_dm_flujo}
\end{figure}
%----------------------------------------------------------------------------------------

\section{Arquitectura del \textit{pipeline}}
\label{sec:arquitectura}

En esta sección se detalla el diseño arquitectónico del sistema desarrollado.
La solución fue estructurada como un flujo de trabajo lineal con
retroalimentación de control cerrado, concebido para transformar flujos de
datos estocásticos y asíncronos provenientes de video y sensores en
recomendaciones biomecánicas accionables y deterministas. La arquitectura
modular permitió el desacoplamiento estricto de las capas de percepción,
modelado físico y razonamiento algorítmico, lo que facilitó la actualización
independiente de los modelos de redes neuronales o los parámetros del algoritmo
genético sin comprometer la integridad operativa del sistema global.

El flujo lógico de datos, conceptualizado en el diagrama de bloques de la
Figura \ref{fig:arquitectura_sistema}, se organiza en cinco etapas secuenciales
críticas que abordan desde la física de la adquisición de la señal hasta la
convergencia matemática de la solución óptima:

\begin{figure}[H]
  \centering
  \includegraphics[width=0.9\textwidth]{Figures/Arquitectura_Pipeline.pdf}
  \caption{Diagrama en bloques detallado de la arquitectura del \textit{pipeline} y el flujo de datos multimodal.}
  \label{fig:arquitectura_sistema}
\end{figure}

\begin{enumerate}
  \item Entrada de datos (capa de adquisición): módulo de abstracción de hardware
        encargado de la captura sincronizada de fuentes heterogéneas, que gestionó
        \textit{buffers} circulares para video de alta velocidad y telemetría mediante 
        protocolos inalámbricos ANT+ \citep{ANTPlus} y BLE (Bluetooth Low Energy) \citep{Gomez2012BLE}.
  \item Procesamiento y análisis (capa de percepción): etapa de acondicionamiento de
        señal donde se ejecutó la inferencia de la red neuronal de detección de pose,
        se realizó la alineación temporal mediante correlación cruzada de señales y se
        aplicó filtrado digital selectivo para la reconstrucción cinemática.
  \item Modelado y simulación (capa física): instanciación del gemelo digital del
        ciclista, resolución del problema de dinámica inversa plana y estimación del
        ($C_d A$).
  \item Optimización multi-objetivo (capa de razonamiento): núcleo algorítmico donde un
        algoritmo genético exploró estocásticamente el espacio de configuraciones
        geométricas para resolver el conflicto entre confort biomecánico, eficiencia de
        potencia y resistencia aerodinámica.
  \item Salida y retroalimentación (capa de interfaz): generación del frente de Pareto,
        visualización sobre el flujo de video original y cierre del ciclo de control
        tras la verificación del ajuste físico.
\end{enumerate}

Esta arquitectura fue diseñada para ejecutarse completamente en el borde, lo
que permitió minimizar la latencia de inferencia y garantizar la privacidad de
los datos biomecánicos del ciclista al mantener todo el procesamiento local sin
necesidad de conexión a servicios externos.

\subsection{Entrada de datos: adquisición multimodal}
Este módulo actuó como la interfaz física con el entorno, lo que permitió
recolectar datos de dominios temporales dispares: el dominio óptico (video) y
el dominio de sensores inerciales/potencia. La sincronización de estos dominios
fue el desafío fundamental de esta etapa.

\subsubsection{Subsistema de video y restricciones ópticas}
La captura de video es el sustrato fundamental para la reconstrucción
cinemática 2D. A diferencia de la grabación convencional, este módulo impuso
restricciones estrictas de hardware para garantizar la integridad de los mapas
de calor generados por la red neuronal en etapas posteriores:

\begin{itemize}
  \item Velocidad de obturación: se impuso un tiempo de exposición máximo de $t_{exp}
          \leq 1/500$ s. Esto fue crítico para eliminar el desenfoque de movimiento
        (\textit{motion blur}) en los marcadores anatómicos (especialmente el maléolo
        lateral \citep{Radiologia2013} y el quinto metatarso \citep{Wikipedia5Meta}), que alcanzaron velocidades angulares máximas
        $\omega_{max}$ en los puntos muertos del ciclo de pedaleo. El desenfoque en
        estas fases introdujo un error sistemático no gaussiano en la inferencia de la
        posición articular.
  \item Tasa de fotogramas: se requirió una frecuencia de muestreo mínima de $f_s \geq
          60$ FPS para cumplir con el teorema de Nyquist-Shannon \citep{Nyquist1928} aplicado a la cinemática
        del pedaleo, donde los armónicos de movimiento significativos se encuentran por
        debajo de 15 Hz.
  \item Normalización de entrada: la resolución de entrada se redimensionó
        dinámicamente para coincidir con el tensor de entrada de la CNN, lo que utilizó
        interpolación bicúbica para evitar artefactos de degradación de la señal
        (\textit{aliasing}).
\end{itemize}

\subsubsection{Entorno de análisis y procesamiento de señales}
El sistema se implementó sobre la plataforma GoldenCheetah \citep{goldencheetah_web}, 
aprovechando su motor de ejecución de Python integrado como núcleo de procesamiento. 
Esta arquitectura permitió la manipulación de datos de alto nivel, lo que superó las 
limitaciones del procesamiento de tramas crudas.

En lugar de calcular la potencia media $\bar{P}$ mediante diferencias de
acumulados a nivel de byte, el sistema utilizó las estructuras de datos
internas de GoldenCheetah, las cuales garantizaron la integridad y
sincronización temporal de las muestras. La implementación se basó en dos
pilares documentados en la API de GoldenCheetah:
\begin{itemize}
  \item Acceso a datos: recuperación de actividades y series temporales mediante
        funciones nativas, lo que aseguró la coherencia de los datos independientemente
        del protocolo de transmisión (ANT+ o BLE).
  \item Procesadores personalizados: implementación de algoritmos definidos por el
        usuario que operaron sobre la señal de potencia $P(t)$ ya depurada, lo que
        permitió el cálculo de métricas derivadas complejas en un entorno interpretado
        de Python.
\end{itemize}

\subsection{Procesamiento y análisis: reconstrucción digital}
Este módulo transformó los datos brutos (imágenes y bytes) en señales
biomecánicas estructuradas, limpias y sincronizadas.

\subsubsection{Detección de pose neuronal}
Se empleó una arquitectura de aprendizaje profundo para inferir la ubicación
probabilística de 33 puntos clave del cuerpo. Tal como se detalló en la sección
\ref{ssec:vision_pose} del marco teórico, se seleccionó el modelo 
\textit{MediaPipe BlazePose} por su alineación con la tarea de seguimiento en 
video y su topología de 33 puntos clave específicos para el análisis biomecánico 
del ciclismo. Para adaptar la red de propósito general al contexto específico del 
ciclismo, se implementó un filtro de Prior Cinemático.

Dado que el ciclismo con calas automáticas constituye una cadena cinética
cerrada (los pies están fijos a los pedales y describen una trayectoria
circular fija), la longitud de los segmentos óseos debe permanecer invariante
en el tiempo. El sistema penalizó y corrigió detecciones que violaron la
antropometría del sujeto:
\begin{equation}
  |L_{seg}(t) - \mu_{L_{seg}}| > \epsilon \Rightarrow \text{Corrección por cinemática inversa}
\end{equation}
Donde $L_{seg}(t)$ es la longitud euclidiana del segmento en el instante $t$ y $\mu_{L_{seg}}$ es la longitud media calibrada.

\subsubsection{Sincronización temporal (algoritmo de correlación de cadencia)}
Un desafío arquitectónico mayor fue alinear el video (muestreado a 60-240 Hz)
con la telemetría (muestreada a 4 Hz), ya que poseen relojes independientes y
latencias de transmisión variables. El \textit{pipeline} implementó un
algoritmo de alineación basado en la cadencia:
\begin{enumerate}
  \item Se extrajo una señal de ``cadencia visual'' ($\omega_{vis}$) al analizar la
        periodicidad de la coordenada vertical de la rodilla ($y_{knee}$) mediante
        Transformada de Fourier de Tiempo Corto (STFT).
  \item Se obtuvo la señal de cadencia del sensor de potencia ($\omega_{sens}$).
  \item Se calculó la correlación cruzada normalizada $R_{vs}(\tau)$ para encontrar el
        desfase óptimo $\tau_{opt}$:
\end{enumerate}
\begin{equation}
  \tau_{opt} = \arg \max_{\tau} \int_{-\infty}^{\infty} \omega_{vis}(t) \cdot \omega_{sens}(t+\tau) \, dt
\end{equation}
Esto permitió alinear el pico de fuerza aplicado en el pedal con el ángulo de la biela correspondiente con precisión de sub-fotograma.

\subsection{Modelado y simulación: física computacional}
En esta etapa se predijo el comportamiento del sistema ciclista-bicicleta bajo
diferentes configuraciones geométricas sin necesidad de intervención física.

\subsubsection{Estimación aerodinámica ($C_d A$)}
El sistema estimó el Área Frontal Proyectada ($A_{proj}$) al utilizar una red
de segmentación semántica sobre la vista lateral del ciclista. Aunque la vista
lateral no capturó directamente el área frontal, se aplicaron modelos de
regresión multivariable (basados en estudios de Bassett y Heil) que
correlacionaron la pose sagital con el área frontal efectiva. El modelo
consideró la altura del torso y el ángulo de apertura de la cadera:
\begin{equation}
  A_{proj} \approx \alpha \cdot H_{torso} \cdot \sin(\theta_{cadera}) + \beta \cdot m_{ciclista}^{0.425} + \gamma
\end{equation}
Esto permitió cuantificar el coste aerodinámico ($P_{aero} \propto v^3 \cdot A_{proj}$) de cambios posturales, como bajar la altura del manillar ($Stack$).

\subsubsection{Dinámica inversa plana}
Al combinar la cinemática filtrada y los vectores de fuerza derivados de la
potencia (se asumió una descomposición de fuerza tangencial efectiva), se
resolvieron las ecuaciones de Newton-Euler para calcular los momentos netos
articulares ($M_{net}$) en tobillo, rodilla y cadera. Esto permitió identificar
ineficiencias mecánicas, como momentos de flexión excesivos en la rodilla que
no contribuyeron a la propulsión.

\subsection{Optimización multi-objetivo: algoritmo genético}
El núcleo decisional utilizó un algoritmo genético adaptado para resolver el
Problema de Optimización Multiobjetivo. Se buscó encontrar el vector de
configuración $\mathbf{x} = [h_{sillin}, x_{sillin}, h_{manillar},
  x_{manillar}]$ que satisfizo simultáneamente objetivos conflictivos.

\subsubsection{Funciones objetivo}
El sistema minimizó tres funciones de coste ($J$):
\begin{enumerate}
  \item Minimizar disconfort ($J_{comfort}$): función de penalización cuadrática que se
        disparó cuando los ángulos articulares $\theta_j$ (extensión de rodilla,
        flexión de hombro, etc.) salieron de los rangos fisiológicos seguros
        predefinidos en la literatura médica:
        \begin{equation}
          J_{comfort}(\mathbf{x}) = \sum_{j=1}^{N} w_j \cdot \max(0, \theta_j(\mathbf{x}) - \theta_{max}, \theta_{min} - \theta_j(\mathbf{x}))^2
        \end{equation}
  \item Maximizar eficiencia biomecánica ($J_{power}$): se buscó minimizar la integral
        del momento articular absoluto necesario para producir una potencia objetivo,
        lo que implicó una palanca más efectiva.
  \item Minimizar resistencia aerodinámica ($J_{aero}$): minimización directa del $C_d
          A$ estimado en la etapa anterior.
\end{enumerate}

\subsubsection{Configuración del algoritmo}
Se empleó una población de $N=100$ individuos con codificación real. Se utilizó
selección por torneo binario, cruzamiento simulado con probabilidad $p_c = 0.9$
y mutación polinómica con probabilidad $p_m = 1/L$ (donde $L$ es el número de
variables). El mecanismo de elitismo aseguró la preservación de las mejores
soluciones del frente de Pareto a través de las generaciones.

\subsection{Salida y retroalimentación}
La etapa final sintetizó los resultados matemáticos para el usuario final (el
biomecánico o el ciclista).
\begin{itemize}
  \item Análisis de Pareto: se presentaron soluciones categorizadas (ej. ``opción
        confort'' vs ``opción aero'') extraídas del conjunto óptimo no dominado.
  \item Visualización de resultados: se generó una representación gráfica del frente de
        Pareto con las configuraciones óptimas, lo que permitió al usuario seleccionar
        el ajuste que mejor se adaptó a sus objetivos biomecánicos y de rendimiento.
\end{itemize}

%----------------------------------------------------------------------------------------

\section{Adquisición de datos}
\label{sec:adquisicion}

El módulo de adquisición de datos se diseñó para garantizar la integridad y la
coherencia temporal de la información de entrada, un requisito crítico para la
validez del análisis posterior. El sistema integró dos fuentes de información
principales: la captura óptica y la telemetría de sensores.

Para la captura óptica, se estableció un protocolo de grabación estandarizado.
Se utilizó una cámara de alta resolución que fue ubicada en una posición
lateral estática, ortogonal al plano sagital del ciclista. Esta disposición
minimizó los errores de paralaje inherentes a la proyección 2D, una limitación
que se discute en el capítulo \ref{Chapter2}. El sistema se configuró para
capturar video a una tasa de cuadros constante, suficiente para evitar el
desenfoque de movimiento durante el pedaleo a cadencias altas.

Simultáneamente, se implementó la recolección de datos de rendimiento mediante
el protocolo ANT+. Se desarrolló un \textit{script} de escucha que se conectó a
los sensores presentes en el rodillo de entrenamiento inteligente y en la
bicicleta. Las variables capturadas incluyeron:

\begin{itemize}
  \item Potencia (W).
  \item Cadencia (RPM).
  \item Velocidad virtual (km/h).
  \item Frecuencia cardíaca (BPM).
\end{itemize}

El desafío técnico principal en esta etapa fue la sincronización. Dado que el
video y los sensores operaron con relojes independientes, se implementó un
mecanismo de alineación temporal en el postprocesamiento. Se utilizó la señal
de cadencia como disparador: el inicio del movimiento de las bielas que fue
detectado en el video se alineó con el primer valor positivo de RPM que fue
registrado por los sensores. % La figura \ref{fig:senales_sensores}
% muestra un ejemplo de las señales capturadas y sincronizadas listas para el
% procesamiento.

% \begin{figure}[htbp]
%   \centering
%   % Placeholder para la imagen: ./Figures/grafico_senales.png
%   \framebox{\parbox{0.8\textwidth}{\centering
%       \vspace{2cm}
%       \textbf{[Figura: Gráfico de señales de sensores]} \\
%       \small\textit{Visualización de Potencia y Cadencia sincronizadas en el tiempo.}
%       \vspace{2cm}
%     }}
%   \caption{Visualización de las señales de telemetría sincronizadas durante una sesión de captura.}
%   \label{fig:senales_sensores}
% \end{figure}

%----------------------------------------------------------------------------------------

\section{Modelado cinemático y biomecánico}
\label{sec:modelado_cinematico}

Una vez que fueron extraídas las coordenadas $(x, y)$ de los puntos anatómicos
clave mediante la red neuronal, se procedió al modelado cinemático. El objetivo
de este módulo fue transformar datos posicionales abstractos en métricas
biomecánicas con significado clínico y deportivo.

El modelo geométrico se construyó con base en la cinemática vectorial. Se
definieron vectores que representan los segmentos corporales del ciclista
(fémur, tibia, torso, húmero, radio). A partir de estos vectores, se calcularon
los ángulos articulares dinámicos en cada cuadro del video mediante el producto
escalar. Los ángulos críticos que fueron modelados incluyeron:

\begin{itemize}
  \item Ángulo de extensión de rodilla (KFA): calculado en el punto muerto inferior (PMI). Es el predictor principal de la altura correcta del sillín.
  \item Ángulo de flexión de cadera: evaluado en el punto muerto superior (PMS). Determina la restricción aerodinámica y el riesgo de pinzamiento acetabular.
  \item Ángulo del hombro y codo: indicadores del alcance (\textit{reach}) y la caída (\textit{drop}) del manillar.
  \item Ángulo del tobillo: analizado para evaluar la técnica de pedaleo (\textit{ankling}).
\end{itemize}

% En la figura \ref{fig:keypoints_angulos} se presenta el esquema de los puntos
% articulares que fueron rastreados y los ángulos resultantes. 
Se aplicó un filtro de suavizado (filtro de Butterworth paso bajo) a las series
temporales de los ángulos para eliminar el ruido de alta frecuencia que fue
introducido por la detección cuadro a cuadro (\textit{jitter}).

% \begin{figure}[htbp]
%   \centering
%   % Placeholder para la imagen: ./Figures/esquema_angulos.png
%   \framebox{\parbox{0.6\textwidth}{\centering
%       \vspace{3cm}
%       \textbf{[Figura: Esquema de ángulos]} \\
%       \small\textit{Diagrama del ciclista con keypoints y ángulos superpuestos.}
%       \vspace{3cm}
%     }}
%   \caption{Representación del modelo cinemático: \textit{keypoints} detectados y ángulos articulares calculados.}
%   \label{fig:keypoints_angulos}
% \end{figure}

La validación de estos ángulos se realizó contra rangos de referencia
bibliográficos. La tabla \ref{tab:rangos_biomecanicos} detalla los parámetros
biomecánicos monitoreados y sus ventanas de operación óptimas utilizadas como
restricciones en el sistema.

\begin{table}[htbp]
  \caption{Parámetros biomecánicos y rangos óptimos definidos para el modelo.}
  \label{tab:rangos_biomecanicos}
  \centering
  \begin{tabular}{lcc}
    \toprule
    Parámetro Articular        & Rango Óptimo (Grados)          & Punto del Ciclo   \\ \midrule
    Extensión de Rodilla (KFA) & 135$^{\circ}$ -- 145$^{\circ}$ & PMI               \\
    Flexión de Cadera          & 50$^{\circ}$ -- 60$^{\circ}$   & PMS               \\
    Ángulo de Hombro           & 80$^{\circ}$ -- 90$^{\circ}$   & Estático/Dinámico \\
    Ángulo de Codo             & 150$^{\circ}$ -- 160$^{\circ}$ & Estático/Dinámico \\
    Extensión de Tobillo       & 90$^{\circ}$ -- 100$^{\circ}$  & Promedio          \\ \bottomrule
  \end{tabular}
\end{table}

%----------------------------------------------------------------------------------------

\section{Modelado del sistema}
\label{sec:modelado_sistema}

Esta sección describe el ``cerebro'' del sistema, que fue compuesto por la
integración del modelo de detección de pose y el motor de optimización.

\subsection{Detección de pose}
Para la etapa de percepción, se implementó el modelo \textit{MediaPipe
  BlazePose}. Tal como se justificó en el marco teórico, este modelo ofrece una
topología de 33 puntos que incluye marcadores específicos para los pies,
esenciales para el análisis de ciclismo. La red neuronal procesó el flujo de
video cuadro a cuadro, y se infirió la posición espacial del esqueleto del
ciclista. Se configuró el modelo con parámetros de confianza de detección y
seguimiento superiores a 0.5 para filtrar falsos positivos en fondos complejos.

\subsection{Motor de optimización}
El núcleo de la propuesta es el algoritmo de optimización multi-objetivo. Dado
que la relación entre los ajustes de la bicicleta y la respuesta biomecánica es
no lineal y compleja, se diseñó un Algoritmo Genético (AG). Este enfoque
heurístico permite explorar eficientemente el espacio de soluciones sin
necesidad de conocer la derivada de la función objetivo.

El problema de optimización se formuló de la siguiente manera:
\begin{itemize}
  \item Genes: los parámetros ajustables de la bicicleta (altura del sillín, retroceso
        del sillín, longitud de la potencia, altura del manillar).
  \item Función de aptitud: una función compuesta que busca minimizar el error entre
        los ángulos medidos y los rangos óptimos, maximizar la eficiencia teórica de
        potencia y minimizar el área frontal (estimación aerodinámica).
\end{itemize}

El flujo del algoritmo, comenzó con una población inicial de configuraciones
aleatorias (dentro de límites mecánicos viables). A través de iteraciones
sucesivas de selección, cruce y mutación, el sistema convergió hacia una
configuración que satisfizo las restricciones biomecánicas con el mejor
compromiso de rendimiento.% que se representa en la figura \ref{fig:flujo_optimizacion}, 

% \begin{figure}[htbp]
%   \centering
%   % Placeholder para la imagen: ./Figures/flujo_algoritmo.png
%   \framebox{\parbox{0.7\textwidth}{\centering
%       \vspace{3cm}
%       \textbf{[Figura: Algoritmo Genético]} \\
%       \small\textit{Diagrama de flujo: Población $\rightarrow$ Evaluación $\rightarrow$ Selección $\rightarrow$ Cruce/Mutación.}
%       \vspace{3cm}
%     }}
%   \caption{Diagrama de flujo del algoritmo de optimización biomecánica.}
%   \label{fig:flujo_optimizacion}
% \end{figure}

En la tabla \ref{tab:parametros_optimizacion} se detallan los parámetros de
configuración del algoritmo genético que fueron utilizados para asegurar la
convergencia y evitar óptimos locales.

\begin{table}[htbp]
  \caption{Parámetros de configuración del algoritmo de optimización.}
  \label{tab:parametros_optimizacion}
  \centering
  \begin{tabular}{lc}
    \toprule
    Parámetro del Algoritmo  & Valor / Configuración          \\ \midrule
    Tamaño de la población   & 100 individuos                 \\
    Número de generaciones   & 50                             \\
    Probabilidad de cruce    & 0.8                            \\
    Probabilidad de mutación & 0.2                            \\
    Método de selección      & Torneo                         \\
    Restricciones anatómicas & Límites fisiológicos definidos \\ \bottomrule
  \end{tabular}
\end{table}

%----------------------------------------------------------------------------------------

\section{Reporte de recomendaciones}
\label{sec:reporte}

La etapa final del diseño corresponde a la interfaz de salida. Se desarrolló un
módulo generador de reportes automatizados en formato PDF, que fue diseñado
para traducir los resultados matemáticos de la optimización en instrucciones
comprensibles para el usuario final.

El reporte sintetiza el análisis en tres secciones:
\begin{enumerate}
  \item Diagnóstico actual: presentación visual de la postura inicial del ciclista con
        los ángulos medidos que fueron superpuestos sobre la imagen original,
        destacando en color rojo aquellos valores fuera del rango seguro.
  \item Plan de ejecución: una lista detallada de las modificaciones mecánicas
        sugeridas. Cada instrucción especifica el componente a ajustar, la dirección
        del ajuste y la magnitud en milímetros (ej. ``Subir sillín 5mm'', ``Adelantar
        sillín 2mm'').
  \item Predicción de resultados: una proyección de los nuevos ángulos biomecánicos
        esperados tras realizar los cambios sugeridos.
\end{enumerate}

Este documento sirve como guía para que el ciclista modifique la configuración
de su bicicleta. Tras la ejecución de los cambios, el sistema sugiere reiniciar
el proceso de adquisición de datos para validar las mejoras, lo que cierra el
ciclo de optimización continua. % La figura \ref{fig:ejemplo_reporte} muestra un
% ejemplo del reporte que fue generado por el prototipo.

% \begin{figure}[htbp]
%   \centering
%   % Placeholder para la imagen: ./Figures/ejemplo_reporte.png
%   \framebox{\parbox{0.6\textwidth}{\centering
%     \vspace{3cm}
%     \textbf{[Figura: Ejemplo de Reporte PDF]} \\
%       \small\textit{Muestra del documento final con las recomendaciones.}
%       \vspace{3cm}
%     }}
%   \caption{Ejemplo del reporte final con recomendaciones de ajuste generado por el sistema.}
%   \label{fig:ejemplo_reporte}
% \end{figure}

% --- FIN DEL CAPÍTULO 3 ---