\chapter{Diseño e implementación}
\label{Chapter3}

En este capítulo se presentan el diseño e implementación del sistema de
optimización biomecánica desarrollado en este trabajo. Se describen la
metodología empleada para estructurar el trabajo, la arquitectura del sistema
propuesto y los módulos que lo componen. La exposición se organiza de manera
secuencial, desde la adquisición de datos hasta la generación del reporte final
de recomendaciones.

%----------------------------------------------------------------------------------------
%	METODOLOGÍA
%----------------------------------------------------------------------------------------
\section{Metodología}
\label{sec:metodologia}

% Tono: Pasado (describe lo que SE HIZO en el trabajo).
Para la gestión y ejecución de este trabajo, se adaptó la metodología CRISP-DM
\citep{IBM_SPSS_CRISP_DM} (\textit{Cross-Industry Standard Process for Data
  Mining}). Esta adaptación fue necesaria para alinear las fases estándar de un
trabajo de minería de datos con los objetivos específicos del trabajo de
optimización biomecánica asistida por inteligencia artificial, como se describe
en la sección 8 del documento de planificación \citep{RGv5Goni}.

% Tono: Pasado (el trabajo de adaptación).
El modelo se reestructuró en las siguientes fases principales, que fueron
ajustadas al contexto del trabajo:

\begin{itemize}
  \item Comprensión del negocio y del problema biomecánico.
  \item Comprensión y adquisición de los datos (video, sensores de rendimiento y datos
        antropométricos).
  \item Preparación y preprocesamiento de los datos, que incluyó la extracción de
        pose y el filtrado de señales.
  \item Modelado y simulación biomecánica y aerodinámica.
  \item Evaluación de las soluciones generadas por el optimizador (análisis del frente
        de Pareto).
  \item Despliegue, que fue materializado en la generación del reporte de
        recomendaciones para el ciclista.
\end{itemize}

La figura \ref{fig:crisp_dm_flujo} muestra el diagrama de flujo de la
metodología implementada.

\begin{figure}[H]
  \centering
  % \includegraphics[width=0.8\textwidth]{Imagenes/crisp_dm.png} % Descomentar para usar la imagen real
  % --- Inicio: Placeholder de la figura ---
  \includegraphics[width=0.7\textwidth]{./Figures/crispdm.pdf}
  % --- Fin: Placeholder de la figura ---
  \caption{Diagrama de flujo de la metodología CRISP-DM adaptada para el trabajo.}
  \label{fig:crisp_dm_flujo}
\end{figure}

%----------------------------------------------------------------------------------------
%	ARQUITECTURA (Versión "Mapa General" para evitar redundancia)
%----------------------------------------------------------------------------------------
\section{Arquitectura del \textit{pipeline}}
\label{sec:arquitectura}

En esta sección se detalla el diseño arquitectónico del sistema desarrollado.
La solución fue estructurada como un flujo de trabajo lineal con
retroalimentación de control cerrado, concebido para transformar flujos de
datos estocásticos y asíncronos provenientes de video y sensores en
recomendaciones biomecánicas accionables y deterministas.

La arquitectura modular permitió el desacoplamiento estricto de las capas de
percepción, modelado físico y razonamiento algorítmico, lo que facilitó la
actualización independiente de los modelos de redes neuronales o los parámetros
del algoritmo genético sin comprometer la integridad operativa del
sistema global.

El flujo lógico de datos, conceptualizado en el diagrama de bloques de la
figura \ref{fig:arquitectura_sistema}.
\begin{figure}[H]
  \centering
  \includegraphics[width=0.9\textwidth]{Figures/Arquitectura_Pipeline.pdf}
  \caption{Diagrama en bloques detallado de la arquitectura del \textit{pipeline} y el flujo de datos multimodal.}
  \label{fig:arquitectura_sistema}
\end{figure}

La arquitectura se organiza en cinco etapas secuenciales
críticas que abordan desde la física de la adquisición de la señal hasta la
convergencia matemática de la solución óptima:

\begin{enumerate}
  \item Entrada de datos (capa de adquisición): módulo de abstracción de hardware
        encargado de la captura sincronizada de fuentes heterogéneas, el que gestionó
        \textit{buffers} circulares para video de alta velocidad y telemetría mediante
        protocolos inalámbricos ANT+ \citep{ANTPlus} y BLE (\textit{Bluetooth Low Energy})
        \citep{Gomez2012BLE}.
  \item Procesamiento y análisis (capa de percepción): etapa de acondicionamiento de
        señal donde se ejecutó la inferencia de la red neuronal de detección de pose,
        se realizó la alineación temporal mediante correlación cruzada de señales y se
        aplicó filtrado digital selectivo para la reconstrucción cinemática.
  \item Modelado y simulación (capa física): instanciación del gemelo digital del
        ciclista, resolución del problema de dinámica inversa plana y estimación del
        ($C_d A$).
  \item Optimización multi-objetivo (capa de razonamiento): núcleo algorítmico donde un
        algoritmo genético exploró estocásticamente el espacio de configuraciones
        geométricas para resolver el conflicto entre confort biomecánico, eficiencia de
        potencia y resistencia aerodinámica.
  \item Salida y retroalimentación (capa de interfaz): generación del frente de Pareto,
        visualización sobre el flujo de video original y cierre del ciclo de control
        tras la verificación del ajuste físico.
\end{enumerate}

Esta arquitectura fue diseñada para ejecutarse completamente en el borde, lo
que permitió minimizar la latencia de inferencia y garantizar la privacidad de
los datos biomecánicos del ciclista al mantener todo el procesamiento local sin
necesidad de conexión a servicios externos. A continuación se detallan la
implementación técnica específica de cada módulo.

%----------------------------------------------------------------------------------------
%	ADQUISICIÓN DE DATOS (Aquí movemos los detalles técnicos de HW y Sync)
%----------------------------------------------------------------------------------------
\section{Adquisición de datos}
\label{sec:adquisicion}

El módulo de adquisición de datos se diseñó para garantizar la integridad y la
coherencia temporal de la información de entrada, un requisito crítico para la
validez del análisis posterior. El sistema integró dos fuentes de información
principales: la captura óptica y la telemetría de sensores.

\subsection{Subsistema de video y restricciones ópticas}
La captura de video es el sustrato fundamental para la reconstrucción
cinemática 2D. Se utilizó una cámara de alta resolución ubicada en una posición
lateral estática, ortogonal al plano sagital del ciclista. Esta disposición
minimizó los errores de paralaje inherentes a la proyección 2D, una limitación
que se discute en el capítulo \ref{Chapter2}.

A diferencia de la grabación convencional, este módulo impuso restricciones
estrictas de hardware para garantizar la integridad de los mapas de calor
generados por la red neuronal en etapas posteriores:

\begin{itemize}
  \item Velocidad de obturación: se impuso un tiempo de exposición máximo de $t_{exp}
          \leq 1/500$ s. Esto fue crítico para eliminar el desenfoque de movimiento
        (\textit{motion blur}) en los marcadores anatómicos (especialmente el maléolo
        lateral \citep{Radiologia2013} y el quinto metatarso \citep{Wikipedia5Meta}),
        los que alcanzan velocidades angulares máximas $\omega_{max}$ en los puntos
        muertos del ciclo de pedaleo. El desenfoque en estas fases introdujo un error
        sistemático no gaussiano en la inferencia de la posición articular.
  \item Tasa de fotogramas: se requirió una frecuencia de muestreo mínima de $f_s \geq
          60$ FPS para cumplir con el teorema de Nyquist-Shannon \citep{Nyquist1928}
        aplicado a la cinemática del pedaleo, donde los armónicos de movimiento
        significativos se encuentran por debajo de 15 Hz.
  \item Normalización de entrada: la resolución de entrada se redimensionó
        dinámicamente para coincidir con el tensor de entrada de la CNN mediante
        interpolación bicúbica, lo que evitó artefactos de degradación de la señal
        (\textit{aliasing}).
\end{itemize}

\subsection{Entorno de análisis y telemetría}
Simultáneamente, se implementó la recolección de datos de rendimiento mediante
protocolos inalámbricos ANT+ \citep{ANTPlus} y BLE (\textit{Bluetooth Low Energy})
\citep{Gomez2012BLE}. El sistema se implementó sobre la plataforma
GoldenCheetah \citep{goldencheetah_web}, cuyo motor de ejecución de Python
integrado fue empleado como núcleo de procesamiento.

Esta arquitectura permitió la manipulación de datos de alto nivel y superó las
limitaciones del procesamiento de tramas crudas. En lugar de calcular la
potencia media $\bar{P}$ mediante diferencias de acumulados a nivel de byte, el
sistema utilizó las estructuras de datos internas de GoldenCheetah, lo que
garantizó la integridad y sincronización temporal de las muestras. La
implementación se basó en dos pilares documentados en la API:

\begin{itemize}
  \item Acceso a datos: recuperación de actividades y series temporales mediante
        funciones nativas, lo que aseguró la coherencia de los datos independientemente
        del protocolo de transmisión.
  \item Procesadores personalizados: implementación de algoritmos definidos por el
        usuario que operaron sobre la señal de potencia $P(t)$ ya depurada.
\end{itemize}

Las variables capturadas incluyeron potencia (W), cadencia (RPM), velocidad
virtual (km/h) y frecuencia cardíaca (BPM) y derivadas.

\subsection{Sincronización temporal (algoritmo de correlación)}
El desafío técnico principal en esta etapa fue la sincronización. Dado que el
video (muestreado a 60-240 Hz) y la telemetría (muestreada a 4 Hz) operan con
relojes independientes y latencias de transmisión variables, se implementó un
mecanismo de alineación temporal en el postprocesamiento.

El \textit{pipeline} implementó un algoritmo de alineación basado en la
cadencia para encontrar el desfase óptimo:

\begin{enumerate}
  \item Se extrajo una señal de cadencia visual ($\omega_{vis}$) mediante el análisis
        de la periodicidad de la coordenada vertical de la rodilla ($y_{knee}$) con la
        transformada de Fourier de tiempo corto (STFT).
  \item Se obtuvo la señal de cadencia del sensor de potencia ($\omega_{sens}$).
  \item Se calculó la correlación cruzada normalizada $R_{vs}(\tau)$ para encontrar el
        desfase óptimo $\tau_{opt}$:
\end{enumerate}

\begin{equation}
  \tau_{opt} = \arg \max_{\tau} \int_{-\infty}^{\infty} \omega_{vis}(t) \cdot \omega_{sens}(t+\tau) \, dt
\end{equation}

Esto permitió alinear el pico de fuerza aplicado en el pedal con el ángulo de
la biela correspondiente con precisión de sub-fotograma, lo que superó la
simple alineación por inicio de movimiento.

%----------------------------------------------------------------------------------------
%	MODELADO (Aquí movemos la física: Ángulos, Aero y Dinámica Inversa)
%----------------------------------------------------------------------------------------
\section{Modelado cinemático y biomecánico}
\label{sec:modelado_cinematico}

Una vez extraídas las coordenadas $(x, y)$ de los puntos anatómicos
clave mediante la red neuronal, se procedió al modelado cinemático. El objetivo
de este módulo fue transformar datos posicionales abstractos en métricas
biomecánicas con significado clínico y deportivo, e integrar modelos físicos
avanzados.

%----------------------------------------------------------------------------------------
%	ANÁLISIS CINEMÁTICO
%----------------------------------------------------------------------------------------
\subsection{Análisis cinemático angular}
\label{subsec:cinematica_angular}

Tras la extracción de las coordenadas articulares mediante la red neuronal, el
sistema abordó la reconstrucción cinemática. A diferencia de enfoques
simplistas que calculan ángulos planos entre vectores 2D, este trabajo
implementó una formulación rigurosa basada en los estándares de la Sociedad
Internacional de Biomecánica (ISB) para evitar errores cinemáticos y
singularidades matemáticas \citep{Wu2002_ISB_Part1}.

\subsubsection{Definición de sistemas de coordenadas locales}
Para garantizar la reproducibilidad clínica y la consistencia física, se
construyeron marcos de referencia ortonormales para cada segmento corporal
(fémur y tibia), lo que permitió abandonar la geometría euclidiana simple en
favor del álgebra vectorial tridimensional.

Para el segmento femoral, se definió el origen $O_f$ en el centro articular de
la cadera (HJC). Se construyó el eje longitudinal $\mathbf{Y}_f$ mediante el
vector unitario que conecta el punto medio de los epicóndilos femorales
($\mathbf{M}_{FE}$) con el HJC:

\begin{equation}
  \mathbf{Y}_f = \frac{\mathbf{r}_{HJC} - \mathbf{M}_{FE}}{||\mathbf{r}_{HJC} - \mathbf{M}_{FE}||}
\end{equation}

Para resolver la orientación espacial, se definió un eje auxiliar
$\mathbf{Z}_{temp}$ que une los epicóndilos lateral y medial, lo que permitió
calcular el eje antero-posterior $\mathbf{X}_f$ mediante producto vectorial, lo
que aseguró la ortogonalidad respecto al plano cuasi-frontal:

\begin{equation}
  \mathbf{X}_f = \frac{\mathbf{Y}_f \times \mathbf{Z}_{temp}}{||\mathbf{Y}_f \times \mathbf{Z}_{temp}||}, \quad \quad \mathbf{Z}_f = \mathbf{X}_f \times \mathbf{Y}_f
\end{equation}

De manera análoga, el sistema de coordenadas tibial ($\mathcal{F}_{tibia}$) se
definió con origen en $\mathbf{M}_{FE}$, lo que utilizó el eje transmaleolar
para la orientación distal.

\subsubsection{Cálculo de la matriz de rotación y ángulos de Euler}
La cinemática angular de la rodilla se obtuvo mediante la proyección de la
orientación del marco tibial sobre el espacio vectorial del fémur. Se definieron
${}^G\mathbf{R}_f$ y ${}^G\mathbf{R}_t$ como las matrices de rotación absoluta
de fémur y tibia respecto al sistema global, y la matriz de orientación relativa
${}^f\mathbf{R}_t$ se calculó como:

\begin{equation}
  {}^f\mathbf{R}_t = ({}^G\mathbf{R}_f)^T \cdot {}^G\mathbf{R}_t = \begin{bmatrix} r_{11} & r_{12} & r_{13} \\ r_{21} & r_{22} & r_{23} \\ r_{31} & r_{32} & r_{33} \end{bmatrix}
\end{equation}

Para descomponer esta matriz en métricas clínicamente significativas, se
implementó la secuencia de rotación de Grood \& Suntay (Z-X-Y), estándar ISB
para la rodilla \citep{GroodSuntay1983}. El ángulo de flexión/extensión
($\alpha$), variable crítica para la optimización de la altura del sillín, se
extrajo mediante:

\begin{equation}
  \alpha_{flexion} = \arctan2(-r_{12}, r_{22})
\end{equation}

Esta formulación desacopló matemáticamente la flexión pura de los movimientos
parásitos de abducción y rotación axial, lo que proporcionó una señal de entrada
limpia para el algoritmo genético.

\subsubsection{Estimación de estado mediante filtro de Kalman}
Uno de los avances metodológicos de este sistema respecto a la literatura
convencional fue la sustitución del filtrado determinista (Butterworth)
\citep{Winter2009_Biomechanics} por un estimador estocástico recursivo. Dado que
el modelo de detección de pose introdujo ruido de alta frecuencia (jitter) y que
el filtrado convencional introdujo latencia de fase inaceptable para aplicaciones
en tiempo real, se implementó un filtro de Kalman lineal \citep{ucm:kalman} basado en un modelo
cinemático de aceleración angular constante (CA).

Para cada grado de libertad articular $\theta$, se definió el vector de estado
$\mathbf{x}_k$ en el instante $k$, el que incluyó explícitamente la velocidad
y aceleración angular necesarias para la etapa dinámica posterior:

\begin{equation}
  \mathbf{x}_k = \begin{bmatrix} \theta_k \\ \dot{\theta}_k \\ \ddot{\theta}_k \end{bmatrix}
\end{equation}

El modelo de transición de estado a priori, el que predijo la cinemática en $k$
basándose en $k-1$ y las ecuaciones de movimiento de Newton, se rigió por la
matriz de transición $\mathbf{F}$:

\begin{equation}
  \hat{\mathbf{x}}_{k|k-1} = \mathbf{F} \hat{\mathbf{x}}_{k-1|k-1}
\end{equation}

donde la matriz de transición de estado se define como:

\begin{equation}
  \mathbf{F} = 
  \begin{bmatrix} 
    1 & \Delta t & \dfrac{1}{2}\Delta t^2 \\ 
    0 & 1 & \Delta t \\
    0 & 0 & 1 
  \end{bmatrix}
\end{equation}

La incertidumbre del modelo se gestionó mediante la matriz de covarianza del
ruido del proceso $\mathbf{Q}$, la que fue diseñada para modelar la
sobreaceleración (\textit{jerk}) como un ruido blanco espectral con densidad
$\Phi_s$. Esto permitió que el filtro se adaptara dinámicamente a cambios bruscos
de cadencia sin perder la suavidad de la señal:

\begin{equation}
  \mathbf{Q} = \Phi_s \begin{bmatrix} \frac{\Delta t^5}{20} & \frac{\Delta t^4}{8} & \frac{\Delta t^3}{6} \\ \frac{\Delta t^4}{8} & \frac{\Delta t^3}{3} & \frac{\Delta t^2}{2} \\ \frac{\Delta t^3}{6} & \frac{\Delta t^2}{2} & \Delta t \end{bmatrix}
\end{equation}

Finalmente, la actualización a posteriori corrigió la predicción mediante el uso
de la medición ruidosa de la red neuronal $\mathbf{z}_k$, la que fue ponderada
por la ganancia de Kalman óptima $\mathbf{K}_k$. Esto resultó en vectores
cinemáticos $(\theta, \dot{\theta}, \ddot{\theta})$ suaves y sincronizados en
fase, los que fueron empleados para alimentar el modelo de dinámica inversa.

\subsection{Estimación aerodinámica ($C_d A$)}
En esta etapa se predijo el comportamiento del sistema ciclista-bicicleta bajo
diferentes configuraciones geométricas sin necesidad de intervención física en
túnel de viento. El sistema estimó el área frontal proyectada ($A_{proj}$)
utilizando una red de segmentación semántica sobre la vista lateral del
ciclista.

Aunque la vista lateral no captura directamente el área frontal, se aplicaron
modelos basados en los estudios de Bassett et al.
\citep{Bassett1999_HourRecord} y Heil \citep{Heil2001_BodyMassScaling}, los que
correlacionaron la pose sagital con el área frontal efectiva. El modelo
consideró la altura del torso y el ángulo de apertura de la cadera:

\begin{equation}
  A_{proj} \approx \alpha \cdot H_{torso} \cdot \sin(\theta_{cadera}) + \beta \cdot m_{ciclista}^{0.425} + \gamma
\end{equation}

donde los parámetros del modelo representan:

\begin{itemize}
  \item $\alpha$: coeficiente de contribución postural que pondera el impacto de la
        altura del torso y la apertura de la cadera en el área frontal. Este parámetro
        captura la geometría del ciclista en función de su posición.
  \item $H_{torso}$: altura vertical del torso medida desde la cadera hasta el hombro
        en la vista lateral, expresada en metros.
  \item $\theta_{cadera}$: ángulo de apertura de la cadera formado entre el torso y el
        muslo. El término $\sin(\theta_{cadera})$ modela la proyección del torso sobre
        el plano frontal.
  \item $\beta$: coeficiente antropométrico que relaciona la masa corporal con el área
        frontal mediante una ley de potencia alométrica.
  \item $m_{ciclista}$: masa corporal del ciclista en kilogramos.
  \item El exponente $0.425$ proviene de las leyes de escalamiento alométrico de Heil,
        las que relacionan el área de superficie corporal con la masa mediante la
        fórmula de Du Bois.
  \item $\gamma$: término de sesgo (intercepción) que representa la contribución
        constante de la bicicleta y el equipamiento del ciclista al área frontal total.
\end{itemize}

Esto permitió cuantificar el coste aerodinámico ($P_{aero} \propto v^3 \cdot
  A_{proj}$) de cambios posturales, como bajar la altura del manillar ($Stack$).

%----------------------------------------------------------------------------------------
%	DINÁMICA INVERSA
%----------------------------------------------------------------------------------------
\subsection{Dinámica inversa plana}
\label{subsec:dinamica_inversa}

La evaluación de la eficiencia biomecánica ($J_{power}$) en la función de coste
del optimizador requiere ir más allá de la cinemática pura para analizar las
causas del movimiento (fuerzas y momentos). Dado que el ciclismo es una
actividad restringida predominantemente al plano sagital, se formuló un modelo
de dinámica inversa plana (2D) para calcular los torques netos
articulares.

Se empleó el método de Newton-Euler recursivo \citep{sdycontrol2023newton},
el que resolvió las ecuaciones de movimiento desde el segmento
distal (pie/pedal) hacia el proximal (cadera). Este enfoque permitió propagar
las fuerzas de reacción del suelo medidas por los sensores de potencia a través
de la cadena cinética del ciclista.

\subsubsection{Definición de parámetros inerciales}
Para cada segmento $i$ (pie, pierna, muslo), se modelaron las propiedades
inerciales basadas en tablas antropométricas estándar escaladas a la masa del
sujeto. Se definieron $m_i$ como la masa del segmento, $I_i$ como el momento de
inercia centroidal y $\mathbf{a}_{cm, i} = [a_{ix}, a_{iy}]^T$ como la
aceleración lineal del centro de masas, la que fue derivada de la cinemática
filtrada en la sección anterior.

\subsubsection{Dinámica del segmento distal (pie)}
El análisis inició en la interfaz pedal-pie. Las fuerzas externas
$\mathbf{F}_{pedal} = [F_{px}, F_{py}]^T$ se descompusieron en el sistema
global. Mediante la aplicación de la segunda ley de Newton, se calcularon las
fuerzas de reacción en el tobillo ($\mathbf{F}_{ankle}$):

\begin{equation}
  F_{ankle\_x} = m_{foot} a_{foot\_x} - F_{px}
\end{equation}
\begin{equation}
  F_{ankle\_y} = m_{foot} (a_{foot\_y} + g) - F_{py}
\end{equation}

Para el equilibrio de momentos, se aplicó la ecuación de Euler respecto al
centro de masas (CM) para eliminar el término de aceleración lineal del CM. El
torque neto muscular en el tobillo $M_{ankle}$ se despejó mediante la
consideración de los momentos generados por las fuerzas de reacción y la inercia
rotacional del segmento. El momento inercial y el momento del pedal se expresan
como:

\begin{equation}
  M_{inercial} = I_{foot} \alpha_{foot} - M_{pedal}
\end{equation}

Los momentos generados por las fuerzas de reacción en el tobillo respecto al CM
del pie son:

\begin{equation}
  M_{ankle\_reaction} = -(x_{ankle}-x_{cm}) F_{ankle_y} + (y_{ankle}-y_{cm}) F_{ankle_x}
\end{equation}

Los momentos generados en el pedal respecto al CM del pie son:

\begin{equation}
  M_{pedal\_reaction} = -(x_{pedal}-x_{cm}) F_{py} + (y_{pedal}-y_{cm}) F_{px}
\end{equation}

Finalmente, el torque neto en el tobillo se calcula como:

\begin{equation}
  M_{ankle} = M_{inercial} + M_{ankle\_reaction} + M_{pedal\_reaction}
\end{equation}

\subsubsection{Propagación recursiva (pierna y muslo)}
Siguiendo el principio de acción y reacción (tercera ley de Newton), las
fuerzas y momentos ejercidos por el pie sobre la pierna fueron
$-\mathbf{F}_{ankle}$ y $-M_{ankle}$. Para el segmento de la pierna (shank),
las ecuaciones de equilibrio en la rodilla se formularon como:

\begin{equation}
  \mathbf{F}_{knee} = m_{shank} (\mathbf{a}_{shank} - \mathbf{g}) + \mathbf{F}_{ankle}
\end{equation}

El momento neto en la rodilla $M_{knee}$, variable crítica que el optimizador
buscó minimizar para reducir el riesgo de lesión patelar, se calculó mediante la
suma de los torques inerciales y los producidos por las fuerzas articulares
distales y proximales respecto al CM de la pierna. El momento inercial de la
pierna y la transmisión del momento del tobillo se expresan como:

\begin{equation}
  M_{inercial\_shank} = I_{shank} \alpha_{shank} + M_{ankle}
\end{equation}

Los momentos generados por las fuerzas de reacción en la rodilla respecto al CM
de la pierna son:

\begin{equation}
  M_{knee\_reaction} = -(x_{knee}-x_{cm}) F_{knee_y} + (y_{knee}-y_{cm}) F_{knee_x}
\end{equation}

Los momentos generados en el tobillo respecto al CM
de la pierna son:

\begin{equation}
  M_{ankle\_shank} = (x_{ankle}-x_{cm}) F_{ankle_y} - (y_{ankle}-y_{cm}) F_{ankle_x}
\end{equation}

Finalmente, el momento neto en la rodilla se calcula como:

\begin{equation}
  M_{knee} = M_{inercial\_shank} + M_{knee\_reaction} + M_{ankle\_shank}
\end{equation}

Este proceso se repitió recursivamente para el segmento del muslo para
obtener el torque de cadera $M_{hip}$.

\subsubsection{Cálculo de potencia articular}
Finalmente, la eficiencia biomecánica se cuantificó mediante la potencia
mecánica instantánea $P_{joint}$ generada o absorbida en cada articulación.
Esta métrica integró la salida cinética con la cinemática, y se calculó como el
producto escalar del momento neto y la velocidad angular relativa de la
articulación $\omega_{joint}$:

\begin{equation}
  P_{joint}(t) = M_{joint}(t) \cdot (\dot{\theta}_{proximal}(t) - \dot{\theta}_{distal}(t))
\end{equation}

La integral de esta potencia a lo largo del ciclo de pedaleo constituyó la
función objetivo $J_{power}$ que el algoritmo genético minimizó, lo que permitió
buscar la configuración geométrica que permitió generar los mismos vatios
externos con el menor coste mecánico interno.

La figura \ref{fig:gemelo_digital_biomecanico} presenta una visualización del
gemelo digital biomecánico implementado, donde se ilustran los vectores de
fuerzas de reacción ($\mathbf{F}_{pedal}$), momentos articulares
($M_{ankle}$, $M_{knee}$, $M_{hip}$), velocidades angulares ($\omega_{knee}$)
y el ángulo de flexión ($\alpha$) calculados mediante el método de dinámica
inversa descrito.
\begin{figure}[H]
  \centering
  \includegraphics[width=1.0\textwidth]{Figures/esqueleto.png}
  \caption{Gemelo digital biomecánico del ciclista mostrando el análisis de dinámica inversa plana.}
  \label{fig:gemelo_digital_biomecanico}
\end{figure}

\section{Modelado del sistema de decisión}
\label{sec:modelado_sistema}

Esta sección describe el cerebro del sistema, compuesto por la integración del
modelo de detección de pose inteligente y el motor de optimización matemática.

\subsection{Detección de pose y corrección cinemática}
Para la etapa de percepción, se implementó el modelo \texttt{MediaPipe
  BlazePose}. Tal como se justificó en el marco teórico, este modelo ofrece una
topología de 33 puntos que incluye marcadores específicos para los pies. La red
neuronal procesó el flujo de video cuadro a cuadro, y se infirió la posición
espacial del esqueleto del ciclista mediante la configuración de parámetros de confianza de
detección y seguimiento superiores a 0,5.

Para adaptar la red de propósito general al contexto del ciclismo, se
implementó un filtro de prior cinemático. Dado que el ciclismo con calas
automáticas constituye una cadena cinética cerrada, la longitud de los
segmentos óseos debe permanecer invariante. El sistema penalizó y corrigió
detecciones que violaron la antropometría del sujeto:

\begin{equation}
  |L_{seg}(t) - \mu_{L_{seg}}| > \epsilon \Rightarrow \text{Corrección por cinemática inversa}
\end{equation}

donde $L_{seg}(t)$ es la longitud euclidiana del segmento en el instante $t$ y
$\mu_{L_{seg}}$ es la longitud media calibrada.

\subsection{Motor de optimización multi-objetivo}
 El núcleo de la propuesta es el algoritmo de optimización multi-objetivo. Dado
 que la relación entre los ajustes de la bicicleta y la respuesta biomecánica es
 no lineal y compleja, se diseñó un Algoritmo Genético (AG). Este enfoque
 heurístico permite explorar eficientemente el espacio de soluciones sin
 necesidad de conocer la derivada de la función objetivo.
 El problema de optimización se formuló de la siguiente manera:
 \begin{itemize}
   \item Genes: los parámetros ajustables de la bicicleta (altura del sillín, retroceso
         del sillín, longitud de la potencia, altura del manillar).
   \item Vector de estado: $\mathbf{x} = [h_{sillin}, x_{sillin}, h_{manillar},
           x_{manillar}]$
 \end{itemize}

\subsubsection{Funciones objetivo detalladas}
 El sistema minimizó tres funciones de coste ($J$) que representan el conflicto
 de objetivos:
 \begin{enumerate}
   \item Minimizar disconfort ($J_{comfort}$): función de penalización cuadrática que se
         disparó cuando los ángulos articulares $\theta_j$ (extensión de rodilla,
         flexión de hombro, etc.) salieron de los rangos fisiológicos seguros
         predefinidos en la literatura médica:
         \begin{equation}
           J_{comfort}(\mathbf{x}) = \sum_{j=1}^{N} w_j \cdot \max(0, \theta_j(\mathbf{x}) - \theta_{max}, \theta_{min} - \theta_j(\mathbf{x}))^2
         \end{equation}
   \item Maximizar eficiencia biomecánica ($J_{power}$): se buscó minimizar la integral
         del momento articular absoluto necesario para producir una potencia objetivo,
         lo que implicó la búsqueda de una palanca más efectiva.
   \item Minimizar resistencia aerodinámica ($J_{aero}$): minimización directa del $C_d
           A$ estimado en la etapa de modelado físico (Sección 3.4.2).
 \end{enumerate}

\subsubsection{Configuración y flujo del algoritmo}
 El flujo del algoritmo comenzó con una población inicial de configuraciones
 aleatorias (dentro de límites mecánicos viables). Mediante iteraciones
 sucesivas de selección, cruce y mutación, el sistema convergió hacia una
 configuración que satisfizo las restricciones biomecánicas.
 Se empleó una población de $N=100$ individuos con codificación real. Se utilizó
 selección por torneo binario, cruzamiento simulado con probabilidad $p_c = 0.9$
 y mutación polinómica con probabilidad $p_m = 1/L$ (donde $L$ es el número de
 variables). El mecanismo de elitismo aseguró la preservación de las mejores
 soluciones del frente de Pareto a través de las generaciones.
 En la tabla \ref{tab:parametros_optimizacion} se detallan los parámetros de
 configuración utilizados.
 \begin{table}[htbp]
   \caption{Parámetros de configuración del algoritmo de optimización.}
   \label{tab:parametros_optimizacion}
   \centering
   \begin{tabular}{lc}
     \toprule
     Parámetro del Algoritmo  & Valor / Configuración          \\ \midrule
     Tamaño de la población   & 100 individuos                 \\
     Número de generaciones   & 50                             \\
     Probabilidad de cruce    & 0.9                            \\
     Probabilidad de mutación & $1/L$ (Polinómica)             \\
     Método de selección      & Torneo Binario                 \\
     Restricciones anatómicas & Límites fisiológicos definidos \\ \bottomrule
   \end{tabular}
 \end{table}

%----------------------------------------------------------------------------------------
%	REPORTE
%----------------------------------------------------------------------------------------
\section{Reporte de recomendaciones}
\label{sec:reporte}

 La etapa final del diseño corresponde a la interfaz de salida. Se desarrolló un
 módulo generador de reportes automatizados en formato PDF, diseñado para
 traducir los resultados matemáticos de la optimización en instrucciones
 comprensibles para el usuario final. El reporte sintetiza el análisis en tres
 secciones:
 \begin{enumerate}
   \item Diagnóstico actual: presentación visual de la postura inicial del ciclista con
         los ángulos medidos superpuestos sobre la imagen original, destacando en color
         rojo aquellos valores fuera del rango seguro.
   \item Plan de ejecución: una lista detallada de las modificaciones mecánicas
         sugeridas. Cada instrucción especifica el componente a ajustar, la dirección
         del ajuste y la magnitud en milímetros (por ejemplo, Subir sillín 5mm o
         Adelantar sillín 2mm).
   \item Predicción de resultados: una proyección de los nuevos ángulos biomecánicos
         esperados tras realizar los cambios sugeridos.
 \end{enumerate}
 Este documento sirve como guía para la modificación de la configuración de la
 bicicleta. Tras la ejecución de los cambios, el sistema sugiere reiniciar el
 proceso de adquisición de datos para validar las mejoras, lo que cierra el
 ciclo de optimización continua.

% --- FIN DEL CAPÍTULO 3 ---